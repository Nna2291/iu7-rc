\rc{Математический анализ}{1}

\section{Теоретические вопросы}

\subsection{Определения}

\begin{question}
  Сформулируйте определение окрестности точки $x \in \R$. 
\end{question}
\begin{answer}
  Окрестностью точки $x$ называется любой интервал, содержащий данную точку.
\end{answer}

\begin{question}
  Сформулируйте определение $\varepsilon$-окрестности точки $x \in \R$.
\end{question}
\begin{answer}
  $\varepsilon$-окрестностью точки $x$ называется интервал с центром в точке $x$ и длиной $\varepsilon$. \[
    S(x, \varepsilon) \qquad \text{или} \qquad u_{\varepsilon}(x)
  \] 
\end{answer}

\begin{question}
   Сформулируйте определение окрестности $+\infty$.
\end{question}
\begin{answer}
  Окрестностью $+\infty$ называется любой интервал вида: \[
  S(a, +\infty), a > 0
  \] 
\end{answer}

\begin{question}
   Сформулируйте определение окрестности $-\infty$.
\end{question}
\begin{answer}
  Окрестностью $-\infty$ называется любой интервал вида: \[
  S(-\infty, -a), a > 0
  \] 
\end{answer}

\begin{question}
   Сформулируйте определение окрестности $\infty$.
\end{question}
\begin{answer}
  Окрестностью $\infty$ называется любой интервал вида:
  \begin{align*}
    S(\infty, a) &= S(-\infty, -a) \cup S(a, +\infty) \\
            &= (-\infty, -a) \cup (+\infty, a), a > 0
  \end{align*}
\end{answer}

\begin{question}
  Сформулируйте определение предела последовательности.
\end{question}
\begin{answer}
 Число $a$ называется пределом последовательности $\{x_{n}\}$, если для любого положительного числа $\varepsilon$ найдётся натуральное число $N(\varepsilon)$ такое, что если порядковый номер $n$ члена последовательности станет больше $N(\varepsilon)$, то имеет место неравенство $|x_{n} - a| < \varepsilon$.\[
 \lim_{n \to \infty} = a \iff (\forall \varepsilon > 0)(\exists N(\varepsilon) \in \N)(\forall  n > N(\varepsilon) \implies |x_{n} - a| < \varepsilon) 
 \] 
\end{answer}

\begin{question}
  Сформулируйте определение сходящейся последовательности.
\end{question}
\begin{answer}
  Последовательность, имеющая предел, назыается сходящейся.
\end{answer}

\begin{question}
  Сформулируйте определение ограниченной последовательности.
\end{question}
\begin{answer}
  Последовательность $\{x_{n}\}$ называется ограниченной, если она ограничена и сверху, и снизу, т.е.: \[
  \forall n \in \N \le x_{n} \le M \iff |x_{n}| \le M
  \] 
\end{answer}

\begin{question}
  Сформулируйте определение монотонной последовательности.
\end{question}

\begin{question}
  Сформулируйте определение возрастающей последовательности.
\end{question}
\begin{answer}
  Последовательность чисел $\{x_{n}\}$ называется возрастающей, если каждый последующий член $x_{n+1} > xn, n \in \N$.
\end{answer}

\begin{question}
  Сформулируйте определение убывающей последовательности.
\end{question}
\begin{answer}
  Последовательность чисел $\{x_{n}\}$ называется убывающей, если каждый последующий член $x_{n+1} < x_{n}$.
\end{answer}

\begin{question}
  Сформулируйте определение невозрастающей последовательности.
\end{question}
\begin{answer}
  Последовательность чисел $\{x_{n}\}$ называется невозрастающей, если каждый последующий член $x_{n+1} \le x_{n}$.
\end{answer}

\begin{question}
  Сформулируйте определение неубывающей последовательности.
\end{question}
\begin{answer}
  Последовательность чисел $\{x_{n}\}$ называется неубывающей, если каждый последующий член $x_{n+1} \ge x_{n}$.
\end{answer}

\begin{question}
  Сформулируйте определение фундаментальной последовательности.
\end{question}
\begin{answer}
  Последовательность $\{x_{n}\}$ называется фундаментальной, если для любого $ε > 0 ∃$ свой порядковый номер $N(\varepsilon)$ такой, что при всех $n \ge N(\varepsilon)$ и $m \ge N(ε)$ выполнено неравенство $|x_{n} − x_{m}| < ε$. 
\end{answer}

\begin{question}
  Сформулируйте критерий Коши существования предела последовательности.
\end{question}
\begin{answer}
  Для того, чтобы последовательность была сходящейся, необходимо и достаточно она была фундаментальной.
\end{answer}

\begin{question}
  Сформулируйте определение по Гейне предела функции.
\end{question}
\begin{answer}
  Число a называется пределом $y=f(x)$ в точке $x0$, если эта функция определена в окрестности точки $a$ и $∀$ последовательнсти $\{x_{n}\}$ из области определения этой функции, сходящейся к $x0$ соответствующая последовательность функций $\{f(x_{n})\}$ сходится к $a$. \[
    \lim_{x \to x_0} f(x) = a \iff (\forall x_{n} \in D_f)(\lim_{n \to \infty} x_{n} = x_{0} \implies \lim_{n \to \infty} f(x_{n}) = a)
\] 
\end{answer}

\begin{question}
  Сформулируйте определение бесконечно малой функции при $x \to x0$. 
\end{question}
\begin{answer}
  Функция называется бесконечно малой при $x \to x0$, если предел функции в этой точке равен 0.
\end{answer}

\begin{question}
  Сформулируйте определение бесконечно большой функции.
\end{question}
\begin{answer}
  Функция $y=f(x)$ называется бесконечно большой функцией, если: \[
  \lim_{x \to x_0} f(x) = \infty
  \] 
\end{answer}

\begin{question}
  Сформулируйте определение бесконечно малых функций одного порядка.
\end{question}

\begin{question}
  Сформулируйте определение несравнимых бесконечно малых функций. 
\end{question}

\begin{question}
  Сформулируйте определение эквивалентных бесконечно малых функций. 
\end{question}

\begin{question}
  Сформулируйте определение порядка малости одной функции относительно
другой.
\end{question}

\begin{question}
  Сформулируйте определение приращения функции. 
\end{question}

\begin{question}
  Cформулируйте определение непрерывности функции в точке (любое). 
\end{question}

\begin{question}
  Сформулируйте определение непрерывности функции на интервале.
\end{question}

\begin{question}
  Сформулируйте определение непрерывности функции на отрезке.
\end{question}

\begin{question}
  Сформулируйте опредление точки разрыва.
\end{question}

\begin{question}
  Сформулируйте определение точки устранимого разрыва.
\end{question}

\begin{question}
  Сформулируйте определение точки разрыва I рода.
\end{question}

\begin{question}
  Сформулируйте определение точки разрыва II рода.
\end{question}


\subsection{Определние предела по Коши}

\begin{question}
  Сформулируйте определение по Коши $\lim_{x \to 0} f(x) = b$, где $b \in \R$.
Приведите соответствующий пример (с геометрической иллюстрацией).
\end{question}

\begin{question}
  Сформулируйте определение по Коши $\lim_{x \to a} = +\infty$, где $a \in \R$.
Приведите соответствующий пример (с геометрической иллюстрацией).
\end{question}

\begin{question}
  СФормулируйте определние по Коши $\lim_{x \to \infty} f(x) = 0$.
Приведите соответствующий пример (с геометрической иллюстрацией).
\end{question}
  
\begin{question}
  СФормулируйте определние по Коши $\lim_{x \to a-0} f(x) = -\infty$, где $a \in \R$.
Приведите соответствующий пример (с геометрической иллюстрацией).
\end{question}

\subsection{Формулировка теорем}

\begin{question}
  Сформулируйте теорему об ограниченности сходящейся числовой последовательности.
\end{question}

\begin{question}
  Сформулируйте теорему о связи функции, ее предела и бесконечно малой.  
\end{question}

\begin{question}
  Сформулируйте теорему о сумме конечного числа бесконечно малых функций.
\end{question}

\begin{question}
  Сформулируйте теорему о произведении бесконечно малой на ограниченную функцию.
\end{question}

\begin{question}
  Сформулируйте теорему о связи бесконечно малой и бесконечно большой функций.
\end{question}

\begin{question}
  Сформулируйте теорему о необходимом и достаточном условии эквивалентности бесконечно малых.
\end{question}

\begin{question}
  Сформулируйте теорему о сумме бесконечно малых разных порядков
\end{question}

