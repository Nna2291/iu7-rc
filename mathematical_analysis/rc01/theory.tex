\rc{Математический анализ}{1}

\section{Теоретические вопросы}

\subsection{Определения}

\begin{question}
  Сформулируйте определение окрестности точки $x \in \R$. 
\end{question}
\begin{answer}
  Окрестностью точки $x$ называется любой интервал, содержащий данную точку.
\end{answer}

\begin{question}
  Сформулируйте определение $\varepsilon$-окрестности точки $x \in \R$.
\end{question}
\begin{answer}
  $\varepsilon$-окрестностью точки $x$ называется интервал с центром в точке $x$ и длиной $2\varepsilon$. \[
    S(x, \varepsilon) \qquad \text{или} \qquad u_{\varepsilon}(x)
  \] 
\end{answer}

\begin{question}
   Сформулируйте определение окрестности $+\infty$.
\end{question}
\begin{answer}
  Окрестностью $+\infty$ называется любой интервал вида: \[
  S(a, +\infty), a > 0
  \] 
\end{answer}

\begin{question}
   Сформулируйте определение окрестности $-\infty$.
\end{question}
\begin{answer}
  Окрестностью $-\infty$ называется любой интервал вида: \[
  S(-\infty, -a), a > 0
  \] 
\end{answer}

\begin{question}
   Сформулируйте определение окрестности $\infty$.
\end{question}
\begin{answer}
  Окрестностью $\infty$ называется любой интервал вида:
  \begin{align*}
    S(\infty, a) &= S(-\infty, -a) \cup S(a, +\infty) \\
            &= (-\infty, -a) \cup (a, +\infty), a > 0
  \end{align*}
\end{answer}

\begin{question}
  Сформулируйте определение предела последовательности.
\end{question}
\begin{answer}
 Число $a$ называется пределом последовательности $\{x_{n}\}$, если для любого положительного числа $\varepsilon$ найдётся натуральное число $N(\varepsilon)$ такое, что если порядковый номер $n$ члена последовательности станет больше $N(\varepsilon)$, то имеет место неравенство $|x_{n} - a| < \varepsilon$.\[
 \lim_{n \to \infty} = a \iff (\forall \varepsilon > 0)(\exists N(\varepsilon) \in \N)(\forall  n > N(\varepsilon) \implies |x_{n} - a| < \varepsilon) 
 \] 
\end{answer}

\begin{question}
  Сформулируйте определение сходящейся последовательности.
\end{question}
\begin{answer}
  Последовательность, имеющая предел, назыается сходящейся.
\end{answer}

\begin{question}
  Сформулируйте определение ограниченной последовательности.
\end{question}
\begin{answer}
  Последовательность $\{x_{n}\}$ называется ограниченной, если она ограничена и сверху, и снизу, т.е.: \[
  \exists M, m : \forall n \in \N \implies m \le x_{n} \le M
  \] 
\end{answer}

\begin{question}
  Сформулируйте определение монотонной последовательности.
\end{question}

\begin{question}
  Сформулируйте определение возрастающей последовательности.
\end{question}
\begin{answer}
  Последовательность чисел $\{x_n\}$ называется возрастающей, если каждый последующий член $x_{n+1} > x_{n}$, $n \in \N$.
\end{answer}

\begin{question}
  Сформулируйте определение убывающей последовательности.
\end{question}
\begin{answer}
  Последовательность чисел $\{x_{n}\}$ называется убывающей, если каждый последующий член $x_{n+1} < x_{n}$.
\end{answer}

\begin{question}
  Сформулируйте определение невозрастающей последовательности.
\end{question}
\begin{answer}
  Последовательность чисел $\{x_{n}\}$ называется невозрастающей, если каждый последующий член $x_{n+1} \le x_{n}$.
\end{answer}

\begin{question}
  Сформулируйте определение неубывающей последовательности.
\end{question}
\begin{answer}
  Последовательность чисел $\{x_{n}\}$ называется неубывающей, если каждый последующий член $x_{n+1} \ge x_{n}$.
\end{answer}

\begin{question}
  Сформулируйте определение фундаментальной последовательности.
\end{question}
\begin{answer}
  Последовательность $\{x_{n}\}$ называется фундаментальной, если для любого $\varepsilon > 0$ $\exists$ свой порядковый номер $N(\varepsilon)$ такой, что при всех $n \ge N(\varepsilon)$ и $m \ge N(\varepsilon)$ выполнено неравенство $|x_{n} - x_{m}| < \varepsilon$. 
\end{answer}

\begin{question}
  Сформулируйте критерий Коши существования предела последовательности.
\end{question}
\begin{answer}
  Для того, чтобы последовательность была сходящейся, необходимо и достаточно чтобы она была фундаментальной.
\end{answer}

\begin{question}
  Сформулируйте определение по Гейне предела функции.
\end{question}
\begin{answer}
  Число a называется пределом $y=f(x)$ в точке $x_0$, если эта функция определена в окрестности точки $a$ и $\forall$ последовательнсти $\{x_n\}$ из области определения этой функции, сходящейся к $x_0$ соответствующая последовательность функций $\{f(x_{n})\}$ сходится к $a$. \[
    \lim_{x \to x_0} f(x) = a \iff (\forall x_{n} \in D_f)(\lim_{n \to \infty} x_{n} = x_{0} \implies \lim_{n \to \infty} f(x_{n}) = a)
\] 
\end{answer}

\begin{question}
  Сформулируйте определение бесконечно малой функции при $x \to x_0$. 
\end{question}
\begin{answer}
  Функция называется бесконечно малой при $x \to x_0$, если: \[
    \lim_{x \to x_0} f(x) = 0
  \] 
\end{answer}

\begin{question}
  Сформулируйте определение бесконечно большой функции.
\end{question}
\begin{answer}
  Функция называется бесконечно большой при $x \to x_0$, если: \[
    \lim_{x \to x_0} f(x) = \infty
  \] 
\end{answer}

\begin{question}
  Сформулируйте определение бесконечно малых функций одного порядка.
\end{question}
\begin{answer}
  Две б.м.ф. $\alpha(x)$ и $\beta(x)$ называются одного порядка малости, если: \[
    \lim_{x \to x_0} \frac{\alpha(x)}{\beta(x)} = const \neq 0
  \] 
\end{answer}

\begin{question}
  Сформулируйте определение несравнимых бесконечно малых функций. 
\end{question}
\begin{answer}
  Две б.м.ф. $\alpha(x)$ и $\beta(x)$ называются \textit{несравнимыми} , если: \[
    \not \exists \lim_{x \to x_0} \frac{\alpha(x)}{\beta(x)}
  \]   
\end{answer}

\begin{question}
  Сформулируйте определение эквивалентных бесконечно малых функций. 
\end{question}
\begin{answer}
  Две б.м.ф. $\alpha(x)$ и $\beta(x)$ называются \textit{эквивалентными} , если: \[
    \lim_{x \to x_0} \frac{\alpha(x)}{\beta(x)} = 1
  \]
\end{answer}

\begin{question}
  Сформулируйте определение порядка малости одной функции относительно
другой.
\end{question}
\begin{answer}
  Б.м.ф. $\alpha(x)$ имеет порядок малости $k$ относительно функции б.м.ф.  $\beta(x)$, если: \[
    \lim_{x \to x_0} \frac{\alpha(x)}{[\beta(x)]^k} = const \neq 0
  \]
\end{answer}

\begin{question}
  Сформулируйте определение приращения функции. 
\end{question}

\begin{question}
  Cформулируйте определение непрерывности функции в точке (любое). 
\end{question}

\begin{question}
  Сформулируйте определение непрерывности функции на интервале.
\end{question}

\begin{question}
  Сформулируйте определение непрерывности функции на отрезке.
\end{question}

\begin{question}
  Сформулируйте опредление точки разрыва.
\end{question}

\begin{question}
  Сформулируйте определение точки устранимого разрыва.
\end{question}

\begin{question}
  Сформулируйте определение точки разрыва I рода.
\end{question}

\begin{question}
  Сформулируйте определение точки разрыва II рода.
\end{question}


\subsection{Определение предела по Коши}

\begin{question}
  Сформулируйте определение по Коши $\lim_{x \to 0} f(x) = b$, где $b \in \R$.
Приведите соответствующий пример (с геометрической иллюстрацией).
\end{question}
\begin{answer}
  Определение:
  \begin{gather*}
    \lim_{x \to 0} f(x) = b \\
    \iff \\
    (\forall \varepsilon > 0)(\exists \delta(\varepsilon) > 0)(\forall x \in \mathring{S}(0, \delta) \implies |f(x) - b| < \varepsilon)
  \end{gather*}

  Пример: \[
    \lim_{x \to 0} (x + b) = b  
  \] 
\end{answer}

\begin{question}
  Сформулируйте определение по Коши $\lim_{x \to a} = +\infty$, где $a \in \R$.
Приведите соответствующий пример (с геометрической иллюстрацией).
\end{question}
\begin{answer}
  Определение:
  \begin{gather*}
    \lim_{x \to a} = +\infty \\
    \iff \\
    (\forall M > 0)(\exists \delta(\varepsilon) > 0)(\forall x \in S(a, \delta) \implies f(x) > M) 
  \end{gather*}
  Пример: \[
  \lim_{x \to a} \frac{1}{|x - a|} = +\infty
  \] 
\end{answer}

\begin{question}
  Сформулируйте определние по Коши $\lim_{x \to \infty} f(x) = 0$.
Приведите соответствующий пример (с геометрической иллюстрацией).
\end{question}
\begin{answer}
  Определение:
  \begin{gather*}
    \lim_{x \to \infty} f(x) = 0 \\
    \iff \\
    (\forall \varepsilon > 0)(\exists N(\varepsilon) > 0)(\forall |x| > N \implies |f(x)| < \varepsilon)
  \end{gather*}
  Пример: \[
    \lim_{x \to \infty} \frac{1}{x} = 0
  \] 
\end{answer}
  
\begin{question}
  Сформулируйте определние по Коши $\lim_{x \to a-0} f(x) = -\infty$, где $a \in \R$.
Приведите соответствующий пример (с геометрической иллюстрацией).
\end{question}
\begin{answer}
  Определение:
  \begin{gather*}
    \lim_{x \to a - 0} f(x) = -\infty \\
    \iff \\
    (\forall M > 0)(\exists \delta(\varepsilon) > 0)(\forall x \in (a - \delta, a) \implies f(x) < -M)
  \end{gather*}

  Пример: \[
    \lim_{x \to a - 0} \frac{1}{x - a} = -\infty
  \] 
\end{answer}


\subsection{Формулировка теорем}

\begin{question}
  Сформулируйте теорему об ограниченности сходящейся числовой последовательности.
\end{question}
\begin{answer}
  Любая сходящаяся последовательность ограничена.
\end{answer}

\begin{question}
  Сформулируйте теорему о связи функции, ее предела и бесконечно малой.  
\end{question}
\begin{answer}
  Функция $y = f(x)$ имеет конечный предел в точке  $x_0$ тогда и только тогда, когда её можно представить в виде суммы предела и некоторой бесконечно малой функции.
\end{answer}

\begin{question}
  Сформулируйте теорему о сумме конечного числа бесконечно малых функций.
\end{question}
\begin{answer}
  Конечная сумма бесконечно малых функции есть бесконечно малая функция.
\end{answer}

\begin{question}
  Сформулируйте теорему о произведении бесконечно малой на ограниченную функцию.
\end{question}
\begin{answer}
  Произведение бесконечно малой функции на ограниченную есть величина бесконечно малая.  
\end{answer}

\begin{question}
  Сформулируйте теорему о связи бесконечно малой и бесконечно большой функций.
\end{question}
\begin{answer}
  Если $\alpha(x)$ - бесконечно большая функция при $x \to x_0$, то $\frac{1}{\alpha(x)}$ - бесконечно малая функция при $x \to x_0$.  
\end{answer}

\begin{question}
  Сформулируйте теорему о необходимом и достаточном условии эквивалентности бесконечно малых.
\end{question}
\begin{answer}
  Две функции $\alpha(x)$ и $\beta(x)$ эквивалентны тогда и только тогда, когда их разность имеет более высокий порядок малости по сравнению с каждой из них.
\end{answer}

\begin{question}
  Сформулируйте теорему о сумме бесконечно малых разных порядков
\end{question}
\begin{answer}
  Сумма бесконечно малых функций разных порядком малости эквивалентно слагаемому низшего порядка малости.
\end{answer}


